\documentclass[12pt]{article}
\pagestyle{myheadings}
\def\ls{\vskip 7pt}                            %One line space.%
\textwidth=6.5in
\textheight=9.0in
\hoffset=0.0in
\voffset=0.0in
\setlength{\topmargin}{-0.5in}
\setlength{\textheight}{9.0in} 
\setlength{\textwidth}{6.5in} 
\setlength{\oddsidemargin}{-0.2in}
\setlength{\evensidemargin}{-0.2in}
%
\def\hfb          {\hfill\break}
\def\hsp          {\hskip 5pt}
\def\vs           {\vskip 10pt}
\def\mybase       {\baselineskip= 13pt plus 1pt minus 1pt}
\def\undertext#1  {$\underline{\smash{\hbox{#1}}}$}
\def\vp           {\hfill\vskip 12pt plus 1pt minus 1pt}

\parskip=10pt
\parindent=0pt

\pagestyle{empty}

\begin{document}


\centerline{\bf PHYS 20323/60323: Fall 2019}
\medskip
\centerline{\bf Assignment \#5}%\#2}
\medskip
\centerline{\bf Due: Thursday Oct. 10, 2019}
\vskip 0.15in

{\bf NOTE: Please show all work so that I  have the opportunity to properly
evaluate your work, or to award partial credit.}

%\noindent {\bf PHOTOMETRY}
\begin{enumerate}


\item {\it The quadratic equation}

\begin{enumerate}
\item Write a program that takes as input three numbers, $a$, $b$, and $c$, and prints out the two solutions to the quadratic equation $ax^2 + bx + c = 0$ using the standard
formula:
\begin{equation}
x = \frac{-b \pm \sqrt{b^2 - 4ac}}{2a}
\end{equation}
Use your program to compute the solutions of $0.001x^2 + 1000x + 0.001 = 0$.\\


\item There is another way to write the solutions to a quadratic equation. Multiplying top and bottom of the solution above by $-b \mp \sqrt{b^2 - 4ac}$, show that the solutions
can also be written as:
\begin{equation}
x = \frac{2c}{-b \mp \sqrt{b^2 - 4ac}}
\end{equation}
Add further lines to your program to print these values in addition to the earlier ones and again use the program to solve $0.001x^2 + 1000x + 0.001 = 0$. What do you see? How do you explain it?\\

\item Using what you have learned, write a new program that calculates both roots of a quadratic equation accurately in all cases.

\end{enumerate}

{\bf For full credit:} upload your program to Github and turn in your answers to
part (b) and a printout of it in action, showing the solution of the equation $0.001x^2 + 1000x + 0.001 = 0$.\\

This is a good example of how computers don’t always work the way you expect them to. If you simply apply the standard formula for the quadratic equation, the computer will sometimes get the wrong answer. In practice the method you have worked out here is the correct way to solve a quadratic equation on a computer, even though it’s more complicated than the standard formula. If you were writing a program that involved solving many quadratic equations this method might be a good candidate for a user- defined function: you could put the details of the solution method inside a function to save yourself the trouble of going through it step by step every time you have a new equation to solve.


\item {\it The semi-empirical mass formula}

In nuclear physics, the semi-empirical mass formula is a formula for
calculating the approximate nuclear binding energy $B$ of an atomic
nucleus with atomic number $Z$ and mass number $A$. The formula looks
like this:

\begin{equation}
B = a_1 A - a_2 A^{2/3} - a_3 \frac{Z^2}{A^{1/3}} - a_4 \frac{(A - 2Z)^2}{A} - \frac{a_5}{A^{1/2}} 
\end{equation}

where, in units of millions of electron volts (MeV), the constants are $a_1 =
15.67$, $a_2 = 17.23$, $a_3 = 0.75$, $a_4 = 93.2$, and

%\begin{equation}
\[ a_5 \left\{ \begin{array} {r@{\quad\tt if \quad}l} 0 & A \;{\tt is
      \; odd}, \\
    12.0 & A \;{\tt and}\; Z \;{\tt are \;both \;even}, \\ -12.0 & A \;{\tt is
     \;  even \; and}\;  Z \;{\tt is
  \;  odd.} \end{array} \right. \]
%\end{equation}

Write a program that takes as its input the values of $A$ and $Z$, and
prints out (a) the binding energy $B$ for the corresponding atom and (b)
the binding energy per nucleon, which is $B/A$. 

Use your program to find
the binding energy of an atom with $A = 58$ and $Z = 28$. (Hint: The
correct answer is around 490 MeV.) 
Also run,  $A = 59$ and $Z = 28$ and $A = 58$ and $Z = 27$.

{\bf For full credit:} Upload your program to github plus an output of the program in action showing the answers it produces.



%\end{enumerate}
\end{enumerate}



\end{document}


Implement all three version of e-X

Implement Orbit height code using function

Due Friday upload to github.










%
%
%1. A common way to describe the usefulness of a telescope and detector as an imager
%is by $A \Omega$, where $A$ is the area of the collecting aperture and $\Omega$ is the field of
%view (in deg$^2$) of the imaging system.  Determine the $A \Omega$ for the following
%systems accessible to (or soon to be accessible to) UVa astronomers:
%
%\begin{enumerate}
%
%\item The Fan Mountain Observatory 1-m and CCD camera.
%
%\item The Bok telescope and new wide field imager.
%
%\item The Blanco telescope and Mosaic camera.
%
%\item The Sloan telescope and imaging camera.
%
%\end{enumerate}
%
%
%2. Several large imaging surveys of the sky have been conducted or are planned.  
%Compare their $A \Omega$:
%
%\begin{enumerate}
%
%\item Sloan Digital Sky Survey
%
%\item 2MASS
%
%\item PanStarrs
%
%\item Large Synoptic Survey Telescope
%
%\end{enumaerate}
%
%
%
%
%4. Seidel aberrations in a 200 inch plate.   
%
%
%5. Explain why alt-az telescopes cannot pactically observe the zenith.
%

% Undersized secondary mirrors for infrared -- why?  cold stop of sky compared to warm stop of mirror.









\end{document}




